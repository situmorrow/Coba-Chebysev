\documentclass{article}
\usepackage{amsmath}
\usepackage{amssymb}
\usepackage{amsthm}
\usepackage{bm}
\usepackage{graphicx}

\begin{document}

\section*{Model Kapal}

\begin{align*}
    \dot{v}' &= a'_{11}v' + a'_{12}r' + b'_1\delta' \\
    \dot{r}' &= a'_{21}v' + a'_{22}r' + b'_2\delta' \\
    \dot{\psi}' &= r' \\
    \dot{x}' &= u'_0 \cos\psi' - v'\sin\psi' \\
    \dot{y}' &= u'_0 \sin\psi' + v'\cos\psi'
\end{align*}
Persamaan ini dapat ditulis menjadi
\begin{align*}
    \dot{\mathbf{S}} = f(\mathbf{S},\mathbf{u})
\end{align*}
Dengan $S = \left[ \begin{matrix}
    v' & r' & \psi' & x' & y'
\end{matrix} \right]^T$. Untuk persamaan model pengukuran output diberikan sebagai berikut

\[
\mathbf{h} = \mathbf{C}\mathbf{S}
\]
dengan

\[
\mathbf{C} = 
\begin{bmatrix}
0 & 0 & 1 & 0 & 0 \\
0 & 0 & 0 & 1 & 0 \\
0 & 0 & 0 & 0 & 1
\end{bmatrix}
\]
Output sistem adalah sudut \textit{yaw}($\psi$), posisi kapal pada sumbu-x ($x$), dan posisi kapal pada sumbu-y ($y$).
\\
Lalu horizon waktu $[t_0, t_0 + T]$ diubah ke domain Chebyshev $\tau \in [-1, 1]$ dengan
\[
\tau = \frac{2(t - t_0)}{T} - 1 
\qquad \Longleftrightarrow \qquad 
t = t_0 + \frac{T}{2}(\tau + 1).
\]
Turunkan $t$ terhadap $\tau$:
\[
\frac{dt}{d\tau} = \frac{T}{2}.
\]
Dengan menggunakan aturan rantai untuk turunan terhadap $\tau$, diperoleh
\[
\frac{dS}{d\tau} = \frac{dS}{dt} \frac{dt}{d\tau} 
= f(S(\tau), u(\tau)) \cdot \frac{T}{2}.
\]
Sehingga persamaan model $\dot{\mathbf{S}} = f(\mathbf{S},\mathbf{u})$ menjadi $\bold{\dot{S}(\tau) = \frac{T}{2}f(S(\tau),u(\tau))}$. \\

Lalu \textbf{S} dan \textbf{U} diaproksimasi dengan interpolasi polinomial Langrange, yaitu
\begin{align}
    S(\tau) = \sum_{i=0}^{N} S_i \, \phi_i(\tau) \label{eq.S_tau}\\
    u(\tau) = \sum_{i=0}^{N} u_i \, \phi_i(\tau) \label{eq.u_tau}
\end{align}
Dengan N adalah order polinomial dan $\phi_i$ adalah basis fungsi langrange.
\begin{align}
    \phi_i(\tau) = \prod_{\substack{j=0 \\ j \ne i}}^{N}\frac{\tau - \tau_j}{\tau_i - \tau_j} \label{eq.phi_tau}
\end{align}
Dari persamaan \eqref{eq.S_tau} diturunkan terhadap $\tau$, sehingga menjadi
\[
\dot{S}(\tau) = \sum_{i=0}^{N} S_i \frac{d\phi_i(\tau)}{d\tau}
\]
Lalu persamaan \eqref{eq.phi_tau} terhadap $\tau$, sehingga menjadi
\[
\phi_i'(\tau) = \phi_i(\tau) 
\sum_{j \ne i} \frac{1}{\tau - \tau_j}
\]
Sehingga diperoleh 
\begin{align*}
    \dot{S}(\tau) = \sum_{i=0}^{N} S_i \phi_i(\tau) \left( 
    \sum_{\substack{j=0 \\ j \ne i}}^{N} \frac{1}{\tau - \tau_j} 
    \right)
\end{align*}
Untuk $k = 1, 2, \ldots, N$ pilih titik Chebyshev-Gauss-Lobatto sebagai titik interpolasi yang diberikan oleh
\[
\tau_k = \cos(\frac{\pi k }{N})
\]

\section*{Design NMPC}
\begin{align*}
J(k) = \sum_{i=1}^{N_p} 
\left\| \mathbf{h}_{\text{ref}}(k+i|k) - \mathbf{h}(k+i|k) \right\|_{Q_{(i)}}^{2}
+ 
\left\| \Delta \mathbf{u}(k+i-1|k) \right\|_{R_{(i)}}^{2}
\end{align*}
dengan kendala

\begin{align*}
\mathbf{s}(k+i|k) &= \mathbf{f}_d\big(\mathbf{s}(k+i-1|k), \mathbf{u}(k+i-1|k)\big) \\
\mathbf{h}(k+i|k) &= \mathbf{C}\mathbf{s}(k+i|k) \\
\mathbf{s}_{\min} &\leq \mathbf{s}(k+i|k) \leq \mathbf{s}_{\max} \\
\mathbf{u}_{\min} &\leq \mathbf{u}(k+i-1|k) \leq \mathbf{u}_{\max} \\
\Delta \mathbf{u}_{\min} &\leq \Delta \mathbf{u}(k+i-1|k) \leq \Delta \mathbf{u}_{\max}
\end{align*}
untuk $i = 1, 2, \ldots, N_p$. Dengan $\Delta \mathbf{u}(k) = \mathbf{u}(k) - \mathbf{u}(k-1)$ 

\subsection*{Dengan Chebysev}
Karena $\Delta \mathbf{u}(k) = \mathbf{u}(k) - \mathbf{u}(k-1)$ dapat diaproksimasi menjadi turunan $\mathbf{u}$ yaitu berdasarkan \eqref{eq.u_tau}, diperoleh

\[
u(\tau) = \sum_{i=0}^{N} u_i \, \phi_i(\tau), 
\qquad \tau \in [-1, 1].
\]
Turunan terhadap waktu:

\begin{align*}
    \dot{u}(t) = \frac{du}{dt} 
= \frac{2}{T} \frac{du}{d\tau} \\
\left.\frac{du}{d\tau}\right|_{\tau_k} 
= \sum_{i=0}^{N} D_{ki} u_i,
\end{align*}
Sehingga fungsi tujuan menjadi:

\begin{align*}
J = \frac{T}{2} \sum_{k=0}^{N} w_k 
\left[ 
\left\| h(x_k) - h_d(\tau_k) \right\|_{Q}^{2} 
+ 
\left\| \frac{2}{T} \sum_{i=0}^{N} D_{ki} u_i \right\|_{R}^{2}
\right]
\end{align*}
Dengan kendala
\begin{align*}
    \frac{2}{T} \sum_{i=0}^{N} D_{ki} s_i &= \frac{T}{2}f(S(\tau), u(\tau)) \\
    h_k &= C s_k \\
    s_{\min} \leq &s_k \leq s_{\max} \\
    u_{\min} \leq &u_k \leq u_{\max} \\
    \dot{u}_{\min} \leq \frac{2}{T} &\sum_{i=0}^{N} D_{ki} u_i \leq \dot{u}_{\max} \\
\end{align*}
untuk $k = 0, \ldots, N$. Dengan $D_{ki}$ adalah evaluasi turunan pada titik kolokasi $\tau = \tau_k$ menghasilkan elemen dari matriks diferensiasi Chebyshev–Gauss–Lobatto (CGL):
\begin{equation}
    D_{ki} = \phi_i'(\tau_k).
\end{equation}

Matriks diferensiasi $D_{ki}$ dapat dinyatakan secara eksplisit sebagai:
\begin{equation}
D_{ki} = 
\begin{cases}
\dfrac{c_k}{c_i}\dfrac{(-1)^{k+i}}{\tau_k - \tau_i}, & k \neq i, \\[8pt]
-\dfrac{\tau_k}{2(1-\tau_k^2)}, & 1 \le k \le N-1, \\[8pt]
\dfrac{2N^2 + 1}{6}, & k = 0, \\[6pt]
-\dfrac{2N^2 + 1}{6}, & k = N,
\end{cases}
\label{eq:Dki}
\end{equation}
dengan $c_0 = c_N = 2$ dan $c_i = 1$ untuk $i = 1, 2, \ldots, N-1$.

\[
\phi_i'(\tau)=\phi_i(\tau)\sum_{j\ne i}\frac{1}{\tau-\tau_j}.
\]

\[
D_{ki}=\phi_i'(\tau_k)=
\begin{cases}
\dfrac{c_k}{c_i}\dfrac{(-1)^{k+i}}{\tau_k-\tau_i}, & k\ne i,\\[6pt]
-\dfrac{\tau_k}{2(1-\tau_k^2)}, & 1\le k\le N-1,\\[6pt]
\dfrac{2N^2+1}{6}, & k=0,\\[6pt]
-\dfrac{2N^2+1}{6}, & k=N,
\end{cases}
\quad c_0=c_N=2,\; c_i=1.
\]

\end{document}