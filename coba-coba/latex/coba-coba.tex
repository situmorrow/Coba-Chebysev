\documentclass{article}
\usepackage{amsmath}
\usepackage{amssymb}
\usepackage{amsthm}
\usepackage{enumitem}
\usepackage{tabularx}
\usepackage{lmodern}
\usepackage{booktabs}
\usepackage{longtable}
\usepackage{geometry}
\usepackage{float}
\usepackage{bm}
\usepackage{graphicx}
\usepackage{physics}
\usepackage{subcaption}
\graphicspath{{../gambar/}}

\begin{document}
\section{Bimbingan}
\subsection{Metodologi}
\subsubsection{Pengkajian Model Kapal}
Pada tahap ini akan menghasilkan model kapal tak linier, yaitu:
\[
\mathbf{\dot{s}} = \mathbf{f}(\mathbf{s},u)
\]
dan model pengukuran
\[
\mathbf{h=Cs}
\]
\subsubsection{Pengubahan domain waktu menjadi domain standar}
Pada tahap ini, waktu yang awalnya pada $[t_0,t_f]$, diubah menjadi domain standar $\tau \in [-1,1]$. Hubungan antara waktu sebenarnya dan waktu terstandar diberikan oleh transformasi linear:
\[
\tau = \frac{2(t-t_0)}{t_f-t_0}-1
\]
Sehingga model kapal yang sebelumnya masih berdomain waktu akan diubah menjadi domain standar $\tau$, dengan 
\[
\tau = \frac{2(t - t_0)}{t_f-t_0} - 1 
\qquad \Longleftrightarrow \qquad 
t = t_0 + \frac{t_f-t_0}{2}(\tau + 1).
\]
Turunkan $t$ terhadap $\tau$:
\[
\frac{dt}{d\tau} = \frac{t_f-t_0}{2}.
\]
Dengan menggunakan aturan rantai untuk turunan terhadap $\tau$, diperoleh
\[
\frac{d\mathbf{s}}{d\tau} = \frac{d\mathbf{s}}{dt} \frac{dt}{d\tau} 
= \mathbf{f}(\mathbf{s}(\tau), u(\tau)) \cdot \frac{t_f-t_0}{2}.
\]
Sehingga persamaan model $\dot{\mathbf{s}} = \mathbf{f}(\mathbf{s},u)$ menjadi $\mathbf{\dot{s}}(\tau) = \frac{t_f-t_0}{2}\mathbf{f}(\mathbf{s}(\tau),u(\tau))$.

\subsubsection{Aproksimasi fungsi dengan polinomial Langrange}
Fungsi $s(\tau)$ dapat diaproksimasi dengan \textbf{polinomial Lagrange\textsuperscript{(2)}} yaitu
\[
s(\tau) = \sum_{i=0}^{N} s_i \, \phi_i(\tau) \label{eq.S_tau}
\]
Dengan N adalah order polinomial $\phi_i$ adalah basis fungsi Lagrange yaitu
\[
\phi_i(\tau) = \prod_{\substack{j=0 \\ j \ne i}}^{N}\frac{\tau - \tau_j}{\tau_i - \tau_j}
\]
\begin{flushleft}
\textsuperscript{(2)}Pendekatan dengan menggunakan fungsi Lagrange dilakukan karena fungsi basis Lagrange secara langsung memastikan nilai aproksimasi melewati setiap titik kolokasi yang telah ditentukan.
\end{flushleft}




\subsubsection{Kolokasi di Titik Chebyshev-Gauss-Lobatto}
Disinilah inti dari Metode Chebyshev Pseudospectral, sebab titik kolokasi yang digunakan adalah titik Chebyshev–Gauss–Lobatto, yaitu:
\[
\tau_i = \cos\left( \frac{i\pi}{N} \right), \quad i = 0, 1, \ldots, N.
\]
Dengan persaman model adalah
\[
\dot{\mathbf{s}}(\tau_i) = \frac{t_f-t_0}{2}\mathbf{f}(\mathbf{s}(\tau_i),u(\tau_i))
\]
\subsubsection{Diferensiasi via Matriks Chebyshev}
\[
\dot{\mathbf{s}}(\tau_i) \approx \sum_{j=0}^{N} D_{ij} \mathbf{s}(\tau_j)
\]
Elemen \textbf{Chebyshev differentiation matrix} $D$ untuk node CGL:
\[
D_{ij} =
\begin{cases}
    \dfrac{c_i (-1)^{i+j}}{c_j (\tau_i - \tau_j)}, & i \neq j, \\[8pt]
    -\dfrac{\tau_j}{2(1 - \tau_j^2)}, & i = j, \; j = 1, \ldots, N-1, \\[10pt]
    \dfrac{2N^2 + 1}{6}, & i = j = 0, \\[6pt]
    -\dfrac{2N^2 + 1}{6}, & i = j = N,
\end{cases}
\]
dengan $c_0 = c_N = 2$, dan $c_j = 1$ untuk lainnya.\\
Sehingga model kapal menjadi
\[
\sum_{j=0}^{N} D_{ij} \mathbf{s}(\tau_j)= \frac{t_f-t_0}{2}\mathbf{f}(\mathbf{s}(\tau_i),u(\tau_i))
\]

\subsubsection{Kuantifikasi Integral (Quadrature) NMPC Secara Umum}
Integral pada $[-1,1]$ diganti jumlah berbobot (aturan Chebyshev--Gauss--Lobatto):
\[
\int_{-1}^{1} g(\tau)\, d\tau \approx \sum_{i=0}^{N} w_i g(\tau_i).
\]
Sehingga fungsi tujuan yang awalnya adalah 
\[
J = \int_{t_0}^{t_f} L(x(t), u(t))\, dt
\]
menjadi:
\[
J \approx \frac{t_f - t_0}{2} \sum_{i=0}^{N} w_i L(x(\tau_i), u(\tau_i)).
\]

\subsubsection{Chebyshev NMPC}
Fungsi tujuan NMPC yaitu:
\begin{align*}
J = \int_{t_0}^{t_f} 
\left(
\| h_{\mathrm{ref}}(t) - h(t) \|_{Q(t)}^2 
+ \| u(t) \|_{R(t)}^2 + \phi_{\mathrm{COLREGS}}
\right) 
dt
\end{align*}
dengan kendala
\begin{align*}
\mathbf{\dot{s}} &= \mathbf{f}(\mathbf{s},u) \\
\mathbf{h} &= C\mathbf{s} \\
\mathbf{s}_{\min} &\leq \mathbf{s}(t) \leq \mathbf{s}_{\max} \\
\mathbf{u}_{\min} &\leq \mathbf{u}(t) \leq \mathbf{u}_{\max} \\
\Delta \mathbf{u}_{\min} &\leq \Delta \mathbf{u}(t) \leq \Delta \mathbf{u}_{\max}
\end{align*}
Dengan $\Delta \mathbf{u}(t) = \mathbf{u}(t) - \mathbf{u}(t-1)$

Karena $\Delta \mathbf{u}(t)$ merupakan perubahan input terhadap waktu, maka perubahan input dapat dituliskan sebagai turunan terhadap waktu, yaitu
\[
\Delta \mathbf{u}(t) = \mathbf{u}(t) - \mathbf{u}(t-1) \approx \dot{\mathbf{u}}(t)
\]
Dengan cara yang sama dengan untuk diferensiasi state, maka turunan input dapat diaproksimasi dengan matriks diferensiasi Chebyshev, yaitu
\[
\dot{\mathbf{u}}(\tau_i) \approx \sum_{j=0}^{N} D_{ij} \mathbf{u}(\tau_j)
\]
Sehingga fungsi tujuan Chebyshev NMPC menjadi:
\[
J = \frac{T}{2} 
\sum_{i=0}^{N} w_i 
\left[
\| h_{\mathrm{ref},i} - h_i \|_Q^2 
+ \| u_i \|_R^2 
+ \phi_{\mathrm{COLREGS},i}
\right]
\]
dengan kendala
\begin{align*}
    \frac{2}{T} \sum_{j=0}^{N} D_{kj} s_j &= f(s_k, u_k) \\
    h_k &= C s_k \\
    s_{\min} \le &s_k \le s_{\max}\\
    u_{\min} \le &u_k \le u_{\max} \\
    \dot{u}_{\min} \le \frac{2}{T} &\sum_{j=0}^{N} D_{kj} u_j \le \dot{u}_{\max} \\
\end{align*}
untuk $k = 0, \ldots, N$. Untuk $N$ genap, $w_k$ adalah
\[
w_0 = w_N = \frac{1}{N^2 - 1},
\]

\[
w_k = \frac{2}{N} 
\left[
\gamma(0,k) + \gamma\left(\frac{N}{2},k\right) 
+ 2 \sum_{i=1}^{N/2 - 1} \gamma(i,k)
\right],
\]
untuk $k = 1, \ldots, N - 1$, dan untuk $N$  ganjil adalah

\[
w_0 = w_N = \frac{1}{N^2},
\]

\[
w_k = \frac{2}{N} 
\left[
\gamma(0,k) + 2 \sum_{i=1}^{(N-1)/2} \gamma(i,k)
\right],
\]
dengan

\[
\gamma(i,k) = \frac{1}{1 - 4i^2} \cos \qty(\frac{2\pi i k}{N}).
\]

\newpage
\subsection{Apa itu metode pseudospectral Method}
Pseudospectral Method (PSM) adalah metode numerik untuk menyelesaikan persamaan diferensial dengan mengaproksimasi fungsi menggunakan polinomial ortogonal (biasanya Chebyshev atau Legendre).

Berbeda dengan metode numerik konvensional (seperti Euler atau Runge-Kutta) yang mendiskritisasi waktu secara seragam, PSM menggunakan \textbf{titik kolokasi nonuniform}\textsuperscript{(1)} yang mengikuti distribusi akar atau ekstrem dari polinomial ortogonal. Tujuannya adalah untuk mencapai akurasi sangat tinggi (eksponensial) dengan jumlah titik yang jauh lebih sedikit.

\textbf{PSM tidak menghitung turunan secara simbolik, tetapi mendefinisikan turunan fungsi pada sejumlah titik kolokasi} menggunakan matriks diferensiasi yang dibangun dari basis polinomial. Dengan cara ini, PSM mengubah persamaan diferensial menjadi sistem aljabar yang dapat diselesaikan dengan metode numerik standar.

\begin{flushleft}
\textsuperscript{(1)} Titik kolokasi adalah titik-titik tertentu dalam domain waktu (atau ruang) di mana persamaan diferensial dianggap “benar” secara eksak.
\end{flushleft}
Kalau begitu apa bedanya dengan diskritisasi? \\
jawab: Diskritisasi adalah proses \textbf{membagi domain waktu} menjadi titik-titik diskret untuk menghitung solusi numerik lokal, sedangkan kolokasi adalah \textbf{memilih titik diskret khusus} untuk memaksa persamaan diferensial “benar” pada titik-titik itu, lalu membangun solusi global berbasis polinomial.

Disinilah letak mengapa diperlukan PSM pada NMPC, sebab PSM membuat NMPC lebih cepat karena ia mengganti integrasi langkah demi langkah dengan aproksimasi polinomial global berorde tinggi yang langsung bisa dimasukkan ke dalam optimasi nonlinear. 

\subsection{Polinomial Chebyshev}
Polinomial Chebyshev adalah dua deret polinomial ortogonal yang berhubungan dengan fungsi kosinus dan sinus, dinotasikan sebagai $T_n(x)$ dan $U_n(x)$. Dengan
\[
T_n(\cos (\theta)) = \cos (n\theta)
\]
\[
U_n(\cos(\theta))\sin\theta = \sin((n+1)\theta)
\]
Artinya jika $x = \cos\theta$, maka
\[
T_0 = 1, \quad T_1=\cos(\theta), \quad T_2=\cos(2\theta), \quad T_3=\cos(3\theta)
\]
Dengan contoh polinomial pertama yaitu:
\begin{table}[H]
    \centering
    \begin{tabular}{c c}
        \toprule
        $n$ & $T_n(x)$ \\
        \midrule
        0 & $1$ \\
        1 & $x$ \\
        2 & $2x^2 - 1$ \\
        3 & $4x^3 - 3x$ \\
        4 & $8x^4 - 8x^2 + 1$ \\
        \bottomrule
    \end{tabular}
\end{table}

\subsection{Chebyshev Pseudospectral Method}
Metode Chebyshev Pseudospectral adalah salah satu dari PSM yang menggunakan Chebyshev–Gauss–Lobatto (CGL) sebegai titik-titik kolokasi. Titik ini berasal dari ekstrem (maksimum dan minimum) polinomial Chebyshev $T_n(x)$. Berikut adalah langkah-langkah memperoleh Chebyshev–Gauss–Lobatto (CGL).
Definisi dasar Chebyshev:
\[
T_n(x) = \cos(n\theta), \qquad x = \cos(\theta)
\]
Langkah 1: Cari titik ekstrem (maksimum dan minimum) \\
Titik ekstrem diperoleh saat turunan terhadap $x$ bernilai nol:
\[
\frac{dT_n(x)}{dx} = 0
\]
Karena $x = \cos\theta$, gunakan turunan rantai:
\[
\frac{dT_n(x)}{dx} = 
\frac{dT_n}{d\theta} \cdot \frac{d\theta}{dx}
\]
Kita tahu:
\[
T_n(x) = \cos(n\theta)
\quad \Rightarrow \quad 
\frac{dT_n}{d\theta} = -n \sin(n\theta)
\]
dan
\[
\frac{d\theta}{dx} = -\frac{1}{\sqrt{1 - x^2}}
\]
Sehingga:
\[
\frac{dT_n}{dx} = 
\frac{n \sin(n\theta)}{\sqrt{1 - x^2}}
\]
Langkah 2: Syarat ekstrem → nolkan turunan \\

\[
\frac{dT_n}{dx} = 0 
\quad \Rightarrow \quad 
\sin(n\theta) = 0
\]
Solusi:
\[
n\theta_i = i\pi, 
\qquad i = 0, 1, 2, \ldots, n
\]

\[
\Rightarrow \quad 
\theta_i = \frac{i\pi}{n}
\]
Langkah 3: Kembalikan ke $x$ \\

\[
x_i = \cos(\theta_i) 
= \cos\left(\frac{i\pi}{n}\right),
\qquad i = 0, 1, \ldots, n
\]
Nah, inilah rumus \textbf{Chebyshev--Gauss--Lobatto nodes.}

Disebut Chebyshev--Gauss--Lobatto karena berasal teori quadrature (integrasi numerik):
\begin{enumerate}
    \item Gauss quadrature = metode integrasi numerik berbasis akar polinomial ortogonal (misal, Chebyshev atau Legendre).
    \item Lobatto = modifikasi Gauss quadrature yang juga menyertakan titik ujung domain (-1 dan +1).
\end{enumerate}

\subsection{Polinomial Lagrange}
Polinomial Lagrange adalah metode interpolasi polinomial yang digunakan untuk membangun polinomial yang melewati sekumpulan titik data tertentu. Misalkan terdapat $N+1$ titik data $(x_0, y_0), (x_1, y_1), \ldots, (x_N, y_N)$, maka polinomial Lagrange $P(x)$ yang melewati semua titik tersebut dapat dituliskan sebagai:
\[
P(x) = \sum_{i=0}^{N} y_i \, L_i(x)
\]
dengan basis fungsi Lagrange $L_i(x)$ didefinisikan sebagai:
\[
L_i(x) = \prod_{\substack{j=0 \\ j \ne i}}^{N}\frac{x - x_j}{x_i - x_j}
\]
Dimana $L_i(x)$ memiliki sifat bahwa $L_i(x_j) = \delta_{ij}$ (delta Kronecker), yang berarti $L_i(x)$ bernilai 1 pada titik $x_i$ dan 0 pada semua titik data lainnya. Dengan demikian, polinomial Lagrange memastikan bahwa $P(x_i) = y_i$ untuk setiap titik data.

Disinilah alasan mengapa pada Chebyshev Pseudospectral Method digunakan polinomial Lagrange, yaitu karena \textbf{fungsi basis Lagrange secara langsung memastikan nilai aproksimasi melewati setiap titik kolokasi yang telah ditentukan}, sehingga memudahkan perhitungan nilai fungsi dan turunannya di setiap node tanpa perlu menghitung koefisien ortogonal polinomial Chebyshev secara eksplisit.

\subsection{Matriks Differensiasi $D_{ki}$}
\[
s(\tau) = \sum_{i=0}^{N} s_i \, \phi_i(\tau)
\]
Dengan $N$ adalah orde polinomial dan $\phi_i$ adalah basis fungsi Lagrange yaitu
\[
\phi_i(\tau) = \prod_{\substack{j=0 \\ j \ne i}}^{N}\frac{\tau - \tau_j}{\tau_i - \tau_j}.
\]
Dari persamaan $s(\tau)$ diturunkan terhadap $\tau$, sehingga menjadi
\[
\dot{s}(\tau) = \sum_{i=0}^{N} s_i \frac{d\phi_i(\tau)}{d\tau}.
\]
Matriks diferensiasi $D$ didefinisikan sebagai nilai turunan basis Lagrange pada titik kolokasi, yaitu
\[
D_{ki} = \left.\frac{d\phi_i(\tau)}{d\tau}\right|_{\tau=\tau_k},
\]
sehingga untuk setiap titik kolokasi $\tau_k$ diperoleh
\[
\dot{s}(\tau_k) = \sum_{i=0}^{N} s_i D_{ki}.
\]

Lalu bagaimana mencari $D_{ki}$?\\
Diketahui jika
\[
\phi_i(\tau) = \prod_{\substack{j=0 \\ j \ne i}}^{N}\frac{\tau - \tau_j}{\tau_i - \tau_j}.
\]
Jika menaruh $\tau = \tau_k$, maka akan terdapat dua kemungkinan, yaitu 
\begin{enumerate}
    \item Jika $k \ne i$, maka
    \[
    \phi_i(\tau_k) = \prod_{\substack{j=0 \\ j \ne i}}^{N}\frac{\tau_k - \tau_j}{\tau_i - \tau_j},
    \]
    di dalam produk terdapat faktor dengan $j=k$ sehingga
    \[
    \frac{\tau_k-\tau_k}{\tau_i-\tau_k} = 0
    \]
    dan akibatnya $\phi_i(\tau_k)=0$.
    \item Jika $k = i$, maka
    \[
    \phi_i(\tau_i) = \prod_{\substack{j=0 \\ j \ne i}}^{N}\frac{\tau_i - \tau_j}{\tau_i - \tau_j} = 1.
    \]
\end{enumerate}
Akan dicari $D_{ki} = \dot{\phi_i}(\tau_k)$ untuk $k \ne i$ dan $k = i$.

\textbf{Kasus 1: $k \ne i$}\\
$\phi_i(\tau)$ difaktorkan sedemikian rupa sehingga memisahkan faktor $(\tau-\tau_k)$:
\[
\phi_i(\tau)
= \frac{\tau-\tau_k}{\tau_i-\tau_k}
  \prod_{\substack{j=0 \\ j \ne i,k}}^{N}\frac{\tau - \tau_j}{\tau_i - \tau_j}
=: A(\tau)\,B(\tau),
\]
dengan
\[
A(\tau)=\frac{\tau-\tau_k}{\tau_i-\tau_k}, \qquad
B(\tau)=\prod_{\substack{j=0 \\ j \ne i,k}}^{N}\frac{\tau - \tau_j}{\tau_i - \tau_j}.
\]
Turunan terhadap $\tau$ adalah
\[
\frac{d\phi_i(\tau)}{d\tau}
= A'(\tau)B(\tau) + A(\tau)B'(\tau).
\]
Evaluasi pada $\tau=\tau_k$ memberikan
\[
A(\tau_k)=0, \qquad A'(\tau_k)=\frac{1}{\tau_i-\tau_k},
\]
sehingga
\[
D_{ki} = \dot{\phi_i}(\tau_k)
= \frac{1}{\tau_i-\tau_k}
   \prod_{\substack{j=0 \\ j \ne i,k}}^{N}\frac{\tau_k - \tau_j}{\tau_i - \tau_j},
    \qquad k\ne i.
\]
Inilah elemen off-diagonal dari matriks diferensiasi $D$ untuk titik kolokasi umum.

\textbf{Kasus 2: $k = i$}\\
Untuk elemen diagonal, digunakan aturan turunan produk pada $\phi_i(\tau)$:
\[
\frac{d\phi_i(\tau)}{d\tau}
= \sum_{\substack{m=0 \\ m \ne i}}^{N}
   \left(
   \prod_{\substack{j=0 \\ j \ne i, j \ne m}}^{N}
   \frac{\tau - \tau_j}{\tau_i - \tau_j}
   \cdot
   \frac{1}{\tau_i - \tau_m}
   \right).
\]
Evaluasi pada $\tau = \tau_i$ menghasilkan
\[
D_{ii} = \dot{\phi_i}(\tau_i)
= \sum_{\substack{m=0 \\ m \ne i}}^{N}
   \left(
   \prod_{\substack{j=0 \\ j \ne i, j \ne m}}^{N}
   \frac{\tau_i - \tau_j}{\tau_i - \tau_j}
   \cdot
   \frac{1}{\tau_i - \tau_m}
   \right).
\]
Karena
\[
\prod_{\substack{j=0 \\ j \ne i, j \ne m}}^{N}
\frac{\tau_i - \tau_j}{\tau_i - \tau_j} = 1,
\]
maka
\[
D_{ii} = \sum_{\substack{m=0 \\ m \ne i}}^{N} \frac{1}{\tau_i - \tau_m}.
\]
Dengan demikian, elemen-elemen matriks diferensiasi $D$ dinyatakan sebagai
\[
D_{ki} =
\begin{cases}
\displaystyle
\frac{1}{\tau_i-\tau_k}
\prod_{\substack{j=0 \\ j \ne i,k}}^{N}
\frac{\tau_k - \tau_j}{\tau_i - \tau_j}, & k\ne i,\\[10pt]
\displaystyle
\sum_{\substack{m=0 \\ m \ne i}}^{N} \frac{1}{\tau_i - \tau_m}, & k=i.
\end{cases}
\]

Untuk mempersingkat notasi, didefinisikan terlebih dahulu
\[
d_i = \prod_{\substack{j=0 \\ j \ne i}}^{N} (\tau_i - \tau_j),
\qquad
w_i = \frac{1}{d_i}
= \frac{1}{\displaystyle\prod_{\substack{j=0 \\ j \ne i}}^{N} (\tau_i - \tau_j)},
\]
dengan $w_i$ dikenal sebagai \emph{barycentric weight} pada titik kolokasi $\tau_i$.

Rumus off-diagonal yang diperoleh pada Kasus~1 dapat ditulis ulang sebagai
\[
D_{ki}
= \frac{1}{\tau_i-\tau_k}
\prod_{\substack{j=0 \\ j \ne i,k}}^{N}
   \frac{\tau_k - \tau_j}{\tau_i - \tau_j}
= \frac{d_k}{d_i}\,\frac{1}{\tau_k - \tau_i},
\qquad k\ne i.
\]
Dengan menggunakan $w_i = 1/d_i$, rumus tersebut menjadi sangat ringkas:
\[
\boxed{
D_{ki} = \frac{w_i}{w_k(\tau_k - \tau_i)}, \qquad k \ne i.
}
\]

Untuk elemen diagonal, dari sifat dasar basis Lagrange berlaku
\[
\sum_{i=0}^{N} \phi_i(\tau) \equiv 1
\quad\Rightarrow\quad
\sum_{i=0}^{N} \phi_i'(\tau_k) = 0
\quad\Rightarrow\quad
\sum_{i=0}^{N} D_{ki} = 0,
\]
sehingga
\[
\boxed{
D_{kk} = -\sum_{\substack{i=0 \\ i \ne k}}^{N} D_{ki}.
}
\]
Persamaan ini ekuivalen dengan ekspresi eksplisit pada Kasus~2 dan sering lebih nyaman digunakan dalam implementasi numerik.

\subsubsection*{Matriks diferensiasi Chebyshev--Gauss--Lobatto}

Untuk Chebyshev Pseudospectral Method, titik kolokasi dipilih sebagai titik
Chebyshev--Gauss--Lobatto (CGL)
\[
\tau_i = \cos\left(\frac{i\pi}{N}\right), \qquad i = 0,1,\dots,N.
\]
Pada titik-titik ini barycentric weight $w_i$ memiliki bentuk tertutup yang sangat sederhana:
\[
w_i = \frac{(-1)^i}{c_i}, \qquad
c_i =
\begin{cases}
2, & i = 0,\,N,\\
1, & 1 \le i \le N-1.
\end{cases}
\]
Mensubstitusikan $w_i = (-1)^i/c_i$ ke dalam rumus barycentric $D_{ki} = \dfrac{w_i}{w_k(\tau_k-\tau_i)}$ menghasilkan
\[
D_{ki}
= \frac{c_k}{c_i}\,\frac{(-1)^{k+i}}{\tau_k - \tau_i},
\qquad k \ne i.
\]
Untuk elemen diagonal interior ($1 \le k \le N-1$) diperoleh secara analitik dengan menjumlahkan elemen off-diagonal tanpa k , maka akan diperoleh
\[
D_{kk} = -\frac{\tau_k}{2\left(1-\tau_k^2\right)}, \qquad k=1,2,\dots,N-1,
\]
sedangkan untuk titik ujung digunakan limit khusus sehingga
\[
D_{00} = \frac{2N^2 + 1}{6}, \qquad
D_{NN} = -\frac{2N^2 + 1}{6}.
\]

Dengan demikian, matriks diferensiasi Chebyshev--Gauss--Lobatto dapat dituliskan secara ringkas sebagai
\[
D_{ki} =
\begin{cases}
\displaystyle
\frac{c_k}{c_i}\,\frac{(-1)^{k+i}}{\tau_k - \tau_i},
& k \ne i,\\[10pt]
\displaystyle
-\dfrac{\tau_k}{2\left(1-\tau_k^2\right)},
& 1 \le k \le N-1,\\[10pt]
\displaystyle
\dfrac{2N^2 + 1}{6}, & k = 0,\\[6pt]
\displaystyle
-\dfrac{2N^2 + 1}{6}, & k = N.
\end{cases}
\]

Rumus ini adalah bentuk standar matriks diferensiasi Chebyshev yang digunakan pada Chebyshev Pseudospectral Method dan menjadi dasar dalam diskretisasi turunan terhadap $\tau$ dalam formulasi NMPC berbasis Chebyshev.

\subsection{Lagrange + CGL VS Chebysev}
\begin{table}[h!]
\centering
\renewcommand{\arraystretch}{1.6} % jarak baris lebih longgar
\setlength{\tabcolsep}{8pt}      % jarak kolom lebih lebar
\footnotesize                   % ukuran font sedikit dikecilkan agar muat

\begin{tabular}{p{4cm} p{6cm} p{6cm}}
\toprule
\textbf{Aspek} &
\textbf{Lagrange + CGL (Nodal)} &
\textbf{Chebyshev Murni (Modal $T_n$)} \\
\midrule

Representasi &
Nilai fungsi di titik node &
Koefisien polinomial $a_n$ \\

Variabel dalam NLP &
$s_k, u_k$ langsung &
Perlu transformasi modal ke nodal \\

Matriks diferensiasi &
Dihitung sekali, langsung digunakan &
Dibangun dari $V_d V^{-1}$ \\

Ekspresi kendala &
Sangat mudah (langsung pada node) &
Rumit (harus dikonversi ke domain nodal) \\

Implementasi &
Implementasi sangat sederhana &
Implementasi jauh lebih kompleks \\

Stabilitas numerik &
Sangat baik (CGL optimal) &
Sangat baik (basis ortogonal) \\

Kecepatan komputasi &
Lebih cepat, tidak ada transformasi &
Lebih lambat (modal–nodal transform) \\

Cocok untuk NMPC &
\textbf{Ya, standar industri} &
Tidak cocok untuk NMPC \\

Digunakan di literatur &
GPOPS-II, Chebyshev PS Methods &
Lebih sering pada teori analitik \\

Kelebihan utama &
Kode simpel, langsung kompatibel dengan constraint &
Basis polinomial elegan secara teori \\

Kekurangan utama &
Tidak elegan secara modal &
Kendala dan dinamika lebih sulit ditulis \\

\bottomrule
\end{tabular}

\caption{Perbandingan metode Lagrange + CGL (nodal) vs Chebyshev murni (modal).}
\end{table}



\subsection{Contoh Penggunaan Polinomial Chebysev}
Berikut adalah perbandingan polinomial Chebysev dan ode45 dalam menyelesaikan persamaan differensial $\dot{x}=x$.
\begin{figure}[H] 
    \centering 
    \includegraphics[width=0.7 \textwidth]{Chebyshev vs Ode45.png} 
    \caption{Perbandingan polinomial Chebysev dan ode45} 
    \label{fig:perbandingan}
\end{figure}

\subsection{Fungsi tujuan di NMPC apakah diskrit atau kontinu?}
Fungsi tujuan NMPC pada dasarnya bersifat kontinu. Hanya saja, untuk keperluan komputasi numerik, fungsi tujuan harus dalam bentuk diskrit (sigma). Sumber buku:
\begin{figure}[H]
    \centering
    \begin{subfigure}{0.48\textwidth}
        \centering
        \includegraphics[width=\textwidth]{obj_kontinu.png}
        \caption{Fungsi Objektif Kontinu}
        \label{fig:obj_kontinu}
    \end{subfigure}
    \hfill
    \begin{subfigure}{0.48\textwidth}
        \centering
        \includegraphics[width=\textwidth]{obj_diskrit.png}
        \caption{Fungsi Objektif Diskrit}
        \label{fig:obj_diskrit}
    \end{subfigure}
\end{figure}

\subsection{$\tau$ itu apa dan bagaimana?}
Variabel $\tau$ merupakan waktu terstandar (domain komputasi) yang digunakan pada metode Chebyshev. Semua polinomial Chebyshev didefinisikan dan bersifat ortogonal pada interval [-1, 1], sehingga domain ini menjadi ruang dasar untuk aproksimasi dan kolokasi dalam Chebyshev Pseudospectral Method. Berikut adalah contoh perubahan domain waktu biasa dan $\tau$ pada menggambar fungsi.
\begin{figure}[H]
    \centering
    \includegraphics[width=\textwidth]{waktu_tau_vs_biasa.png}
    \caption{Perbandingan waktu biasa vs $\tau$}
    \label{fig:waktu_tau}
\end{figure}


\subsection{Alasan kenapa polinomial Chebyshev digunakan? Bisa bandingin dengan Laguerre dan Langrange}
Ada beberapa kelebihan dari Polinomial Chebyshev yaitu
\begin{enumerate}
    \item Akurasi tinggi (eksponensial konvergen)
    \item Lebih sedikit titik kolokasi dibanding metode grid uniform
\end{enumerate}
Dibalik kelebihan, terdapat kekurangan dari Polinomial Chebyshev, yaitu
\begin{enumerate}
    \item Kurang cocok untuk horizon sangat panjang (clustering di ujung)
    \item Butuh transformasi domain untuk tiap horizon
\end{enumerate}

Jika dibandingan dengan polinomial legendre dan Laguerre:
\begin{longtable}{p{3.5cm} p{4cm} p{4cm} p{4cm}}
\toprule
\textbf{Aspek} & \textbf{Chebyshev PSM} & \textbf{Legendre PSM} & \textbf{Laguerre PSM} \\
\midrule
\endfirsthead
\multicolumn{4}{c}{{Lanjutan dari halaman sebelumnya}} \\
\toprule
\textbf{Aspek} & \textbf{Chebyshev PSM} & \textbf{Legendre PSM} & \textbf{Laguerre PSM} \\
\midrule
\endhead
\midrule
\endfoot
\bottomrule
\endlastfoot

Basis polinomial &
$T_n(\tau)=\cos(n\arccos\tau)$ &
$L_n(\tau)$ (Legendre polynomial) &
$L_n(x)$ (Laguerre polynomial) \\

Domain definisi &
$[-1,1]$ (finite horizon) &
$[-1,1]$ (finite horizon) &
$[0,\infty)$ (infinite horizon) \\

Titik kolokasi &
Chebyshev–Gauss–Lobatto: $\tau_i=\cos\left(\frac{i\pi}{N}\right)$ (eksplisit) &
Legendre–Gauss–Lobatto: akar $(1-\tau^2)L_N'(\tau)=0$ (implisit) &
Laguerre–Gauss: akar $L_N'(x)=0$ di $[0,\infty)$ \\

Perhitungan node &
Langsung dari fungsi cos &
Perlu solusi numerik akar polinomial &
Perlu solusi numerik + domain tak hingga \\

Bobot integrasi (quadrature) &
Bentuk tertutup, sederhana &
Lebih kompleks, dihitung numerik &
Bobot menurun eksponensial $e^{-x}$ \\

Distribusi titik &
Padat di ujung domain, jarang di tengah &
Merata di tengah &
Padat di dekat nol, jarang di jauh (meluruh) \\

Akurasi &
Eksponensial untuk fungsi halus di domain terbatas &
Eksponensial juga, lebih stabil di tengah &
Eksponensial untuk fungsi yang meluruh di domain tak hingga \\

Kelebihan utama &
Node dan $D$-matrix eksplisit, komputasi cepat &
Simetri dan cocok untuk fungsi interior &
Ideal untuk horizon tak hingga dan fungsi meluruh \\

Kelemahan utama &
Kurang efisien untuk horizon panjang (cluster di tepi) &
Perlu komputasi akar polinomial &
Tidak cocok untuk horizon terbatas, sulit normalisasi waktu \\

Sifat ortogonalitas &
Terhadap bobot $1/\sqrt{1-\tau^2}$ &
Terhadap bobot $1$ &
Terhadap bobot $e^{-x}$ \\

Aplikasi umum &
NMPC, optimasi lintasan kapal/pesawat &
Aerospace dan finite-time benchmark (NASA) &
Infinite-horizon control, adaptif, sistem termal \\

Keterkaitan domain &
Perlu transformasi linear $t \leftrightarrow \tau$ &
Sama, domain tetap terbatas &
Tidak perlu transformasi (semi-tak hingga) \\

Kemudahan implementasi &
Sangat mudah, $O(N^2)$ matrix analitik &
Sedang, butuh akar numerik &
Paling sulit, domain tak hingga dan bobot eksponensial \\

Kesesuaian untuk NMPC &
\textbf{Sangat cocok} untuk finite horizon, cepat dan stabil &
Cocok juga tapi komputasi lebih berat &
Tidak cocok untuk NMPC finite horizon (domain tak hingga) \\

\end{longtable}

Sehingga dapat disimpulkan Metode NMPC berbasis Chebyshev Pseudospectral Method dipilih karena memiliki kemampuan tinggi dalam menyelesaikan permasalahan kontrol optimal pada sistem dengan dinamika nonlinear kompleks seperti kapal, dengan keunggulan utama berupa akurasi tinggi, efisiensi komputasi, dan kestabilan numerik yang sangat baik.

\subsection{Sifat Orthogonalitas berpengaruh dimana?}
Sifat ortogonalitas polinomial Chebyshev menyebabkan setiap basis fungsi bekerja secara independen, sehingga menghindari interferensi antar komponen aproksimasi. Dalam Pseudospectral Method, hal ini meningkatkan kestabilan numerik dan mempercepat konvergensi. Akibatnya, pada implementasi Nonlinear Model Predictive Control (NMPC), prediksi lintasan menjadi lebih akurat dan penyelesaian optimasi lebih efisien.

Secara sederhana, jika menggunakan \textbf{polinomial biasa (non-ortogonal)}, perubahan pada satu koefisien aproksimasi dapat mempengaruhi seluruh fungsi aproksimasi, sehingga memperumit perhitungan dan mengurangi akurasi. Dengan \textbf{polinomial ortogonal} seperti Chebyshev, setiap koefisien hanya mempengaruhi bagian tertentu dari fungsi, sehingga memudahkan optimasi dan meningkatkan performa NMPC.

\subsection{Fungsi COLREGS di Chebyshev}
Fungsi COLREGS adalah fungsi taklinear. Tujuannya adalah memberikan pembobotan pada posisi kapal relatif terhadap halangan. Fungsi COLREGS didefinisikan sebagai berikut
\begin{align*}
    \phi_{\text{COLREGS}} =
&\sum_{i \in \text{GW}} w_{\text{GW}} F_{\text{GW}}\big(p_{\text{rel}}(k + j|k)\big)+ \\
&\sum_{i \in \text{HO}} w_{\text{HO}} F_{\text{HO}}\big(p_{\text{rel}}(k + j|k)\big)
+ \\
&\sum_{i \in \text{OT}} w_{\text{OT}} F_{\text{OT}}\big(p_{\text{rel}}(k + j|k)\big)
\end{align*}
dengan: \\
i : setiap halangan yang terdeteksi \\
$w_{OT}, w_{HO}, w_{GW}$: nilai pembobot \\
$F_{OT}, F_{HO}, F_{GM}$: fungsi lapangan potensial\\
$\mathbf{p}_{rel}(k+j|k)$: prediksi posisi titik tengah kapal relatif pada koordinat sumbu kapal halangan untuk $t=k+j$ pada saat $t=k$.\\

Perhitungan posisi titik tengah kapal relatif terhadap halangan adalah sebagai berikut:
\[
p_{\text{rel}}(k + j|k) = 
R\big(\psi_{\text{TS}}(k + j|k)\big)
\big[
p_{\text{OS}}(k + j|k) - p_{\text{TS}}(k + j|k)
\big]
\]

dengan

\[
R\big(\psi_{\text{TS}}(k + j|k)\big) =
\begin{bmatrix}
\cos\big(\psi_{\text{TS}}(k + j|k)\big) & \sin\big(\psi_{\text{TS}}(k + j|k)\big) \\
-\sin\big(\psi_{\text{TS}}(k + j|k)\big) & \cos\big(\psi_{\text{TS}}(k + j|k)\big)
\end{bmatrix}
\]
dimana: \\
$p_{\text{OS}}(k+j|k)$: posisi titik tengah kapal sendiri (own-ship) \\
$p_{\text{TS}}(k+j|k)$: posisi titik tengah kapal halangan (target-ship). 

Fungsi lapangan potensial $F_{\text{OT}}, F_{\text{HO}}, F_{\text{GW}}$ 
dirancang agar kapal dapat melakukan manuver untuk menghindari tabrakan 
dengan mengacu kepada \textit{COLREGS}. Rancangan fungsi lapangan potensial diberikan sebagai berikut:

\begin{align*}
F_{\text{GW}} &= 0.5 \times f(\alpha_{x1} x_{\text{rel}}(k+j|k)) 
\times \big(1 + f(\alpha_{y1}(y_{\text{rel}}(k+j|k) - y_{0,\text{GW}}))\big) \\
F_{\text{HO}} &= 0.5 \times f(\alpha_{y2} y_{\text{rel}}(k+j|k)) 
\times \big(1 + f(\alpha_{x2}(x_{0,\text{HO}} - x_{\text{rel}}(k+j|k)))\big) \\
F_{\text{OT}} &= 0.5 \times f(\alpha_{x3}(x_{0,\text{OT}} - x_{\text{rel}}(k+j|k))) 
\times f(\alpha_{y3}(y_{\text{rel}}(k+j|k) - y_{0,\text{OT}}))
\end{align*}
dengan $\alpha_{x1}, \alpha_{x2}, \alpha_{x3}, \alpha_{y1}, \alpha_{y2}, \alpha_{y3}$ 
merupakan parameter untuk menentukan kemiringan dari fungsi lapangan potensial.  
Parameter $y_{0,\text{GW}}, x_{0,\text{HO}}, x_{0,\text{OT}}, y_{0,\text{OT}}$ 
menentukan pergeseran untuk menyesuaikan fungsi potensial.

Fungsi $f$ merupakan fungsi taklinier sebagai berikut:
\[
f(x) = \frac{x}{\sqrt{1 + x^2}}.
\]
Untuk gambaran fungsi COLREGS dapat dilihat pada gambar berikut:
\begin{figure}[H]
    \centering
    \includegraphics[width=1 \textwidth]{fungsi_lapangan.png}
    \caption{Gambaran Fungsi COLREGS}
    \label{fig:colregs_field}
\end{figure}
Nilai positif pada fungsi lapangan potensial menunjukkan area bahaya yang harus dihindari oleh kapal sendiri. Sedangkan nilai negatif menunjukkan area aman untuk dilalui kapal sendiri.

Fungsi COLREGS tidak diubah menjadi dalam domain Chebyshev karena fungsi ini hanya berfungsi sebagai penalti pada fungsi tujuan NMPC. Sehingga fungsi COLREGS tetap dalam domain waktu biasa.

\newpage
\section*{Model Kapal}

\begin{align*}
    \dot{v}' &= a'_{11}v' + a'_{12}r' + b'_1\delta' \\
    \dot{r}' &= a'_{21}v' + a'_{22}r' + b'_2\delta' \\
    \dot{\psi}' &= r' \\
    \dot{x}' &= u'_0 \cos\psi' - v'\sin\psi' \\
    \dot{y}' &= u'_0 \sin\psi' + v'\cos\psi'
\end{align*}
Persamaan ini dapat ditulis menjadi
\begin{align*}
    \dot{\mathbf{S}} = f(\mathbf{S},\mathbf{u})
\end{align*}
Dengan $S = \left[ \begin{matrix}
    v' & r' & \psi' & x' & y'
\end{matrix} \right]^T$. Untuk persamaan model pengukuran output diberikan sebagai berikut

\[
\mathbf{h} = \mathbf{C}\mathbf{S}
\]
dengan

\[
\mathbf{C} = 
\begin{bmatrix}
0 & 0 & 1 & 0 & 0 \\
0 & 0 & 0 & 1 & 0 \\
0 & 0 & 0 & 0 & 1
\end{bmatrix}
\]
Output sistem adalah sudut \textit{yaw}($\psi$), posisi kapal pada sumbu-x ($x$), dan posisi kapal pada sumbu-y ($y$).
\\
Lalu horizon waktu $[t_0, t_0 + T]$ diubah ke domain Chebyshev $\tau \in [-1, 1]$ dengan
\[
\tau = \frac{2(t - t_0)}{T} - 1 
\qquad \Longleftrightarrow \qquad 
t = t_0 + \frac{T}{2}(\tau + 1).
\]
Turunkan $t$ terhadap $\tau$:
\[
\frac{dt}{d\tau} = \frac{T}{2}.
\]
Dengan menggunakan aturan rantai untuk turunan terhadap $\tau$, diperoleh
\[
\frac{dS}{d\tau} = \frac{dS}{dt} \frac{dt}{d\tau} 
= f(S(\tau), u(\tau)) \cdot \frac{T}{2}.
\]
Sehingga persamaan model $\dot{\mathbf{S}} = f(\mathbf{S},\mathbf{u})$ menjadi $\bold{\dot{S}(\tau) = \frac{T}{2}f(S(\tau),u(\tau))}$. \\
Lalu \textbf{S} dan \textbf{U} diaproksimasi dengan interpolasi polinomial Langrange, yaitu
\begin{align}
    S(\tau) = \sum_{i=0}^{N} S_i \, \phi_i(\tau) \label{eq.S_tau}\\
    u(\tau) = \sum_{i=0}^{N} u_i \, \phi_i(\tau) \label{eq.u_tau}
\end{align}
Dengan N adalah order polinomial dan $\phi_i$ adalah basis fungsi langrange.
\begin{align}
    \phi_i(\tau) = \prod_{\substack{j=0 \\ j \ne i}}^{N}\frac{\tau - \tau_j}{\tau_i - \tau_j} \label{eq.phi_tau}
\end{align}
Dari persamaan \eqref{eq.S_tau} diturunkan terhadap $\tau$, sehingga menjadi
\[
\dot{S}(\tau) = \sum_{i=0}^{N} S_i \frac{d\phi_i(\tau)}{d\tau}
\]
Lalu persamaan \eqref{eq.phi_tau} terhadap $\tau$, sehingga menjadi
\[
\phi_i'(\tau) = \phi_i(\tau) 
\sum_{j \ne i} \frac{1}{\tau - \tau_j}
\]
Sehingga diperoleh 
\begin{align*}
    \dot{S}(\tau) = \sum_{i=0}^{N} S_i \phi_i(\tau) \left( 
    \sum_{\substack{j=0 \\ j \ne i}}^{N} \frac{1}{\tau_i - \tau_j} 
    \right)
\end{align*}
Untuk $k = 1, 2, \ldots, N$ pilih titik Chebyshev-Gauss-Lobatto sebagai titik interpolasi yang diberikan oleh
\[
\tau_k = \cos(\frac{\pi k }{N})
\]

\section*{Design NMPC}
\subsection*{Objective Function 1}
\begin{align*}
J(k) = \sum_{i=1}^{N_p} 
\left\| \mathbf{h}_{\text{ref}}(k+i|k) - \mathbf{h}(k+i|k) \right\|_{Q_{(i)}}^{2}
+ 
\left\| \Delta \mathbf{u}(k+i-1|k) \right\|_{R_{(i)}}^{2}
\end{align*}
dengan kendala

\begin{align*}
\mathbf{s}(k+i|k) &= \mathbf{f}_d\big(\mathbf{s}(k+i-1|k), \mathbf{u}(k+i-1|k)\big) \\
\mathbf{h}(k+i|k) &= \mathbf{C}\mathbf{s}(k+i|k) \\
\mathbf{s}_{\min} &\leq \mathbf{s}(k+i|k) \leq \mathbf{s}_{\max} \\
\mathbf{u}_{\min} &\leq \mathbf{u}(k+i-1|k) \leq \mathbf{u}_{\max} \\
\Delta \mathbf{u}_{\min} &\leq \Delta \mathbf{u}(k+i-1|k) \leq \Delta \mathbf{u}_{\max}
\end{align*}
untuk $i = 1, 2, \ldots, N_p$. Dengan $\Delta \mathbf{u}(k) = \mathbf{u}(k) - \mathbf{u}(k-1)$ 

\subsection*{Dengan Chebyshev}
Karena $\Delta \mathbf{u}(k) = \mathbf{u}(k) - \mathbf{u}(k-1)$ dapat diaproksimasi menjadi turunan $\mathbf{u}$ yaitu berdasarkan \eqref{eq.u_tau}, diperoleh

\[
u(\tau) = \sum_{i=0}^{N} u_i \, \phi_i(\tau), 
\qquad \tau \in [-1, 1].
\]
Turunan terhadap waktu:

\begin{align*}
    \dot{u}(t) = \frac{du}{dt} 
= \frac{2}{T} \frac{du}{d\tau} \\
\left.\frac{du}{d\tau}\right|_{\tau_k} 
= \sum_{i=0}^{N} D_{ki} u_i,
\end{align*}
Sehingga fungsi tujuan menjadi:

\begin{align*}
J = \frac{T}{2} \sum_{k=0}^{N} w_k 
\left[ 
\left\| h(x_k) - h_d(\tau_k) \right\|_{Q}^{2} 
+ 
\left\| \frac{2}{T} \sum_{i=0}^{N} D_{ki} u_i \right\|_{R}^{2}
\right]
\end{align*}
Dengan kendala
\begin{align*}
    \sum_{i=0}^{N} D_{ki} s_i &= \frac{T}{2}f(S(\tau), u(\tau)) \\
    h_k &= C s_k \\
    s_{\min} \leq &s_k \leq s_{\max} \\
    u_{\min} \leq &u_k \leq u_{\max} \\
    \dot{u}_{\min} \leq \frac{2}{T} &\sum_{i=0}^{N} D_{ki} u_i \leq \dot{u}_{\max} \\
\end{align*}
untuk $k = 0, \ldots, N$. Untuk $N$ genap, $w_k$ adalah
\[
w_0 = w_N = \frac{1}{N^2 - 1},
\]

\[
w_k = \frac{2}{N} 
\left[
\gamma(0,k) + \gamma\left(\frac{N}{2},k\right) 
+ 2 \sum_{i=1}^{N/2 - 1} \gamma(i,k)
\right],
\]
untuk $k = 1, \ldots, N - 1$, dan untuk $N$  ganjil adalah

\[
w_0 = w_N = \frac{1}{N^2},
\]

\[
w_k = \frac{2}{N} 
\left[
\gamma(0,k) + 2 \sum_{i=1}^{(N-1)/2} \gamma(i,k)
\right],
\]
dengan

\[
\gamma(i,k) = \frac{1}{1 - 4i^2} \cos \qty(\frac{2\pi i k}{N}).
\] 
Dan $D_{ki}$ adalah evaluasi turunan pada titik kolokasi $\tau = \tau_k$ menghasilkan elemen dari matriks diferensiasi Chebyshev–Gauss–Lobatto (CGL):
\begin{equation}
    D_{ki} = \phi_i'(\tau_k) \label{eq.DKi}
\end{equation} 
Matriks diferensiasi $D_{ki}$ dapat dinyatakan secara eksplisit sebagai:
\begin{equation}
D_{ki} = 
\begin{cases}
\dfrac{c_k}{c_i}\dfrac{(-1)^{k+i}}{\tau_k - \tau_i}, & k \neq i, \\[8pt]
-\dfrac{\tau_k}{2(1-\tau_k^2)}, & 1 \le k \le N-1, \\[8pt]
\dfrac{2N^2 + 1}{6}, & k = 0, \\[6pt]
-\dfrac{2N^2 + 1}{6}, & k = N,
\end{cases}
\label{eq:Dki}
\end{equation}
dengan $c_0 = c_N = 2$ dan $c_i = 1$ untuk $i = 1, 2, \ldots, N-1$.

\subsection*{Simulasi NMPC dengan CASADI hasil ChatGPT}
\begin{figure}[h]
    \centering
    % ===== Gambar kiri =====
    \begin{subfigure}{0.48\textwidth}
        \centering
        \includegraphics[width=\textwidth]{CHEBYSEV_NMPC.png}
        \caption{Simulasi NMPC dengan Chebyshev}
        \label{fig:Chebyshev}
    \end{subfigure}
    \hfill
    % ===== Gambar kanan =====
    \begin{subfigure}{0.35\textwidth}
        \centering
        \includegraphics[width=\textwidth]{NMPC_BIASA.png}
        \caption{Simulasi NMPC konvensional}
        \label{fig:nmpc}
    \end{subfigure}

    \caption{Perbandingan hasil simulasi antara NMPC Chebyshev dan NMPC biasa.}
    \label{fig:comparison}
\end{figure}

\subsection*{Catatan}
N : Orde polinomial Chebyshev \\
i : indeks indeks titik basis Lagrange (atau node tempat state/variabel disimpan) \\
k : indeks titik kolokasi tempat turunan sistem dipaksakan berlaku

\newpage
\section*{Coret-Coret}
Dengan $s=[\,v' \ r' \ \psi' \ x' \ y'\,]^\top$, kendali $u'=\delta'$, dan kecepatan surge konstan $u'_0$, model takliniernya:
\[
\dot s = f(s,u') \equiv
\begin{bmatrix}
-0.6174\,v' - 0.1036\,r' + 0.01\,u'\\
-5.0967\,v' - 3.4047\,r' + u'\\
r'\\
u'_0\cos\psi' - v'\sin\psi'\\
u'_0\sin\psi' + v'\cos\psi'
\end{bmatrix}.
\]

Ambil titik CGL untuk $N=4$ dan $T=5$:
\[
\tau_0=1,\quad \tau_1=\tfrac{\sqrt2}{2},\quad \tau_2=0,\quad
\tau_3=-\tfrac{\sqrt2}{2},\quad \tau_4=-1,
\]
dan tulis $S_i:=S(\tau_i)=[\,v'_i\ r'_i\ \psi'_i\ x'_i\ y'_i\,]^\top$, $u_i':=u'(\tau_i)$.

\paragraph{Persamaan kolokasi.}
Untuk setiap $k=0,\dots,4$ berlaku
\[
\sum_{i=0}^{4} D_{ki}\,S_i \;=\; \frac{T}{2}\, f(S_k,u'_k)
\;=\; 5\, f(S_k,u'_k),
\]
yang komponen–per–komponen menjadi
\begin{align*}
\sum_{i=0}^{4} D_{ki}\,v'_i &= 5\Big(-0.6174\,v'_k - 0.1036\,r'_k + 0.01\,u'_k\Big),\\
\sum_{i=0}^{4} D_{ki}\,r'_i &= 5\Big(-5.0967\,v'_k - 3.4047\,r'_k + u'_k\Big),\\
\sum_{i=0}^{4} D_{ki}\,\psi'_i &= 5\,r'_k,\\
\sum_{i=0}^{4} D_{ki}\,x'_i &= 5\Big(u'_0\cos\psi'_k - v'_k\sin\psi'_k\Big),\\
\sum_{i=0}^{4} D_{ki}\,y'_i &= 5\Big(u'_0\sin\psi'_k + v'_k\cos\psi'_k\Big).
\end{align*}

\paragraph{Matriks diferensiasi CGL ($N=4$).}
\[
D=\begin{bmatrix}
  5.50000000 & -6.82842712 &  2.00000000 & -1.17157288 &  0.50000000\\
  1.70710678 & -0.70710678 & -1.41421356 &  0.70710678 & -0.29289322\\
 -0.50000000 &  1.41421356 &  0.00000000 & -1.41421356 &  0.50000000\\
  0.29289322 & -0.70710678 &  1.41421356 &  0.70710678 & -1.70710678\\
 -0.50000000 &  1.17157288 & -2.00000000 &  6.82842712 & -5.50000000
\end{bmatrix}.
\]
maka persamaan menjadi

\paragraph{(1) Komponen $v'$:}
\begin{align*}
5.5v'_0 - 6.8284v'_1 + 2v'_2 - 1.1716v'_3 + 0.5v'_4 &= 5(-0.6174v'_0 - 0.1036r'_0 + 0.01u'_0),\\
1.7071v'_0 - 0.7071v'_1 - 1.4142v'_2 + 0.7071v'_3 - 0.2929v'_4 &= 5(-0.6174v'_1 - 0.1036r'_1 + 0.01u'_1),\\
-0.5v'_0 + 1.4142v'_1 - 1.4142v'_3 + 0.5v'_4 &= 5(-0.6174v'_2 - 0.1036r'_2 + 0.01u'_2),\\
0.2929v'_0 - 0.7071v'_1 + 1.4142v'_2 + 0.7071v'_3 - 1.7071v'_4 &= 5(-0.6174v'_3 - 0.1036r'_3 + 0.01u'_3),\\
-0.5v'_0 + 1.1716v'_1 - 2v'_2 + 6.8284v'_3 - 5.5v'_4 &= 5(-0.6174v'_4 - 0.1036r'_4 + 0.01u'_4).
\end{align*}

\paragraph{(2) Komponen $r'$:}
\begin{align*}
5.5r'_0 - 6.8284r'_1 + 2r'_2 - 1.1716r'_3 + 0.5r'_4 &= 5(-5.0967v'_0 - 3.4047r'_0 + u'_0),\\
1.7071r'_0 - 0.7071r'_1 - 1.4142r'_2 + 0.7071r'_3 - 0.2929r'_4 &= 5(-5.0967v'_1 - 3.4047r'_1 + u'_1),\\
-0.5r'_0 + 1.4142r'_1 - 1.4142r'_3 + 0.5r'_4 &= 5(-5.0967v'_2 - 3.4047r'_2 + u'_2),\\
0.2929r'_0 - 0.7071r'_1 + 1.4142r'_2 + 0.7071r'_3 - 1.7071r'_4 &= 5(-5.0967v'_3 - 3.4047r'_3 + u'_3),\\
-0.5r'_0 + 1.1716r'_1 - 2r'_2 + 6.8284r'_3 - 5.5r'_4 &= 5(-5.0967v'_4 - 3.4047r'_4 + u'_4).
\end{align*}

\paragraph{(3) Komponen $\psi'$:}
\begin{align*}
5.5\psi'_0 - 6.8284\psi'_1 + 2\psi'_2 - 1.1716\psi'_3 + 0.5\psi'_4 &= 5r'_0,\\
1.7071\psi'_0 - 0.7071\psi'_1 - 1.4142\psi'_2 + 0.7071\psi'_3 - 0.2929\psi'_4 &= 5r'_1,\\
-0.5\psi'_0 + 1.4142\psi'_1 - 1.4142\psi'_3 + 0.5\psi'_4 &= 5r'_2,\\
0.2929\psi'_0 - 0.7071\psi'_1 + 1.4142\psi'_2 + 0.7071\psi'_3 - 1.7071\psi'_4 &= 5r'_3,\\
-0.5\psi'_0 + 1.1716\psi'_1 - 2\psi'_2 + 6.8284\psi'_3 - 5.5\psi'_4 &= 5r'_4.
\end{align*}

\paragraph{(4) Komponen $x'$:}
\begin{align*}
5.5x'_0 - 6.8284x'_1 + 2x'_2 - 1.1716x'_3 + 0.5x'_4 &= 5(u'_0\cos\psi'_0 - v'_0\sin\psi'_0),\\
1.7071x'_0 - 0.7071x'_1 - 1.4142x'_2 + 0.7071x'_3 - 0.2929x'_4 &= 5(u'_0\cos\psi'_1 - v'_1\sin\psi'_1),\\
-0.5x'_0 + 1.4142x'_1 - 1.4142x'_3 + 0.5x'_4 &= 5(u'_0\cos\psi'_2 - v'_2\sin\psi'_2),\\
0.2929x'_0 - 0.7071x'_1 + 1.4142x'_2 + 0.7071x'_3 - 1.7071x'_4 &= 5(u'_0\cos\psi'_3 - v'_3\sin\psi'_3),\\
-0.5x'_0 + 1.1716x'_1 - 2x'_2 + 6.8284x'_3 - 5.5x'_4 &= 5(u'_0\cos\psi'_4 - v'_4\sin\psi'_4).
\end{align*}

\paragraph{(5) Komponen $y'$:}
\begin{align*}
5.5y'_0 - 6.8284y'_1 + 2y'_2 - 1.1716y'_3 + 0.5y'_4 &= 5(u'_0\sin\psi'_0 + v'_0\cos\psi'_0),\\
1.7071y'_0 - 0.7071y'_1 - 1.4142y'_2 + 0.7071y'_3 - 0.2929y'_4 &= 5(u'_0\sin\psi'_1 + v'_1\cos\psi'_1),\\
-0.5y'_0 + 1.4142y'_1 - 1.4142y'_3 + 0.5y'_4 &= 5(u'_0\sin\psi'_2 + v'_2\cos\psi'_2),\\
0.2929y'_0 - 0.7071y'_1 + 1.4142y'_2 + 0.7071y'_3 - 1.7071y'_4 &= 5(u'_0\sin\psi'_3 + v'_3\cos\psi'_3),\\
-0.5y'_0 + 1.1716y'_1 - 2y'_2 + 6.8284y'_3 - 5.5y'_4 &= 5(u'_0\sin\psi'_4 + v'_4\cos\psi'_4).
\end{align*}




\newpage
\section*{Materi}
\section*{Chebyshev-Gauss-Lobatto (CGL)}
CGL adalah sekumpulan titik kolokasi (collocation points) yang digunakan dalam metode pseudospectral untuk mendiskritisasi fungsi kontinu menjadi bentuk polinomial berderajat N.
CGL merupakan special case dari titik Chebyshev-Gauss (CG), tetapi dengan tambahan dua titik batas $(-1,1)$ sehingga domainnya lengkap di interval $[-1,1]$. Titik ini didefinisikan sebagai:
\[
\tau_k = \cos \left(\frac{\pi k}{N} \right)
\]
Note:
\begin{enumerate}
    \item Titik kolokasi adalah sekumpulan titik diskrit di sepanjang domain waktu atau ruang tempat persamaan diferensial sistem dipaksakan berlaku secara eksak
    \item Metode pseudospectral adalah metode numerik untuk menyelesaikan persamaan diferensial atau masalah kontrol optimal dengan cara: \begin{enumerate}
        \item Mengaproksimasi fungsi keadaan dan kontrol menggunakan polinomial ortogonal orde tinggi (seperti Chebyshev atau Legendre)
        \item Menentukan turunan dan integralnya hanya di titik-titik kolokasi (biasanya titik-titik Chebyshev–Gauss–Lobatto atau Legendre–Gauss–Lobatto)
    \end{enumerate}
\end{enumerate}

\section*{Pembukatian $D_{ki}$}
Kita mulai dari definisi fungsi basis Lagrange untuk titik-titik kolokasi $\{\tau_j\}_{j=0}^N$:
\begin{equation}
    \phi_i(\tau) = \prod_{\substack{j=0 \\ j \ne i}}^{N} \frac{\tau - \tau_j}{\tau_i - \tau_j},
    \qquad
    \phi_i(\tau_k) = \delta_{ik}.
\end{equation}

Turunan terhadap $\tau$ diberikan oleh
\begin{equation}
    \phi_i'(\tau) = \phi_i(\tau) 
    \sum_{\substack{j=0 \\ j \ne i}}^{N} \frac{1}{\tau - \tau_j}.
\end{equation}

Evaluasi pada titik kolokasi $\tau = \tau_k$ memberikan elemen matriks diferensiasi:
\begin{equation}
    D_{ki} = \phi_i'(\tau_k).
\end{equation}

\subsection*{Bentuk Umum dengan Bobot Barisentris}

Untuk sebarang himpunan titik kolokasi, definisikan bobot barisentris sebagai:
\begin{equation}
    w_i = \frac{1}{\displaystyle \prod_{\substack{m=0 \\ m \ne i}}^{N} (\tau_i - \tau_m)}.
\end{equation}

Dengan bobot tersebut, turunan fungsi basis pada titik-titik kolokasi dapat dinyatakan sebagai
\begin{equation}
    D_{ki} =
    \begin{cases}
        \dfrac{w_i}{w_k} \dfrac{1}{\tau_k - \tau_i}, & k \ne i, \\[8pt]
        -\displaystyle\sum_{\substack{m=0 \\ m \ne k}}^{N} D_{km}, & k = i.
    \end{cases}
    \label{eq:Dki_general}
\end{equation}

\subsection*{Khusus untuk Titik Chebyshev--Gauss--Lobatto (CGL)}

Untuk titik-titik Chebyshev-Gauss-Lobatto yang didefinisikan sebagai
\begin{equation}
    \tau_j = \cos\!\left(\frac{\pi j}{N}\right), \qquad j = 0, 1, \dots, N,
\end{equation}
bobot barisentrisnya memiliki bentuk proporsional sebagai berikut:
\begin{equation}
    w_j \propto (-1)^j \, c_j,
    \qquad
    c_0 = c_N = \tfrac{1}{2}, \quad c_j = 1 \text{ untuk } j = 1, \dots, N-1.
\end{equation}

Substitusi ke dalam Persamaan~\eqref{eq:Dki_general} menghasilkan
\begin{equation}
    \boxed{
    D_{ki} = \frac{c_k}{c_i} \frac{(-1)^{k+i}}{\tau_k - \tau_i}, \qquad k \ne i.
    }
    \label{eq:Dki_offdiag}
\end{equation}

Untuk elemen diagonal diperoleh dengan memanfaatkan sifat baris-jumlah-nol
$\sum_{m=0}^N D_{km} = 0$, sehingga
\begin{equation}
    D_{kk} = - \sum_{\substack{m=0 \\ m \ne k}}^{N} D_{km}.
\end{equation}

Setelah disederhanakan menggunakan identitas trigonometri
$\tau_k = \cos\left(\frac{\pi k}{N}\right)$, bentuk eksplisitnya adalah
\begin{equation}
    \boxed{
    D_{kk} =
    \begin{cases}
        -\dfrac{\tau_k}{2(1 - \tau_k^2)}, & 1 \le k \le N-1, \\[8pt]
        \dfrac{2N^2 + 1}{6}, & k = 0, \\[8pt]
        -\dfrac{2N^2 + 1}{6}, & k = N.
    \end{cases}
    }
    \label{eq:Dki_diag}
\end{equation}

\subsection*{Hasil Akhir}

Maka, matriks diferensiasi Chebyshev--Gauss--Lobatto dapat dinyatakan secara lengkap sebagai:
\begin{equation}
    \boxed{
    D_{ki} =
    \begin{cases}
        \dfrac{c_k}{c_i} \dfrac{(-1)^{k+i}}{\tau_k - \tau_i}, & k \ne i, \\[8pt]
        -\dfrac{\tau_k}{2(1 - \tau_k^2)}, & 1 \le k \le N-1, \\[8pt]
        \dfrac{2N^2 + 1}{6}, & k = 0, \\[8pt]
        -\dfrac{2N^2 + 1}{6}, & k = N.
    \end{cases}
    }
\end{equation}

Bentuk inilah yang digunakan sebagai \textit{differentiation matrix} $D_{ki}$ dalam metode
\textit{Chebyshev pseudospectral}, yang merepresentasikan turunan fungsi kontinu terhadap
variabel $\tau$ pada titik-titik kolokasi $\tau_k \in [-1,1]$.
\end{document}